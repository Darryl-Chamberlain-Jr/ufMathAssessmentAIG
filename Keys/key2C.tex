\documentclass{article}[10pt]
\usepackage{multicol, enumerate, enumitem, hyperref, color, soul, setspace, parskip, fancyhdr, amssymb, amsthm, amsmath, bbm, latexsym, units, mathtools}
\everymath{\displaystyle}
\usepackage[headsep=0.5cm,headheight=0cm, left=1 in,right= 1 in,top= 1 in,bottom= 1 in]{geometry}

\begin{document}
This is the Answer Key for Module 2 Version C.

6. First, find the equation of the line containing the two points below. Then, write the equation as $ y=mx+b $ and choose the intervals that contain $m$ and $b$.
$$ (7, -5) \text{ and } (2, 2) $$ 
The solution is $ y = -1.4 x + 4.8 $ 

\begin{enumerate}[label=\Alph*.] 
\item $ m \in [-6, -1] \text{ and } b \in [4.63, 5.66] $ 

 * Correct option. 
\item $ m \in [-6, 2] \text{ and } b \in [-5.25, -4.47] $ 

  Corresponds to using the correct slope and getting the negative y-intercept. 
\item $ m \in [-3, 0] \text{ and } b \in [-12.9, -11.65] $ 

  Corresponds to using the correct slope/equation but not distributing correctly using the first point. 
\item $ m \in [-6, 2] \text{ and } b \in [-0.09, 0.91] $ 

  Corresponds to using the correct slope/equation but not distributing correctly using the second point. 
\item $ m \in [-1, 7] \text{ and } b \in [-1.48, -0.26] $ 

  Corresponds to using the negative slope and the correct equation. 
\end{enumerate} 
 
General Comments: Remember to keep your points in order when plugging in to the slope formula.

-----------------------------------------------

7. Write the equation of the line in the graph below in the form $Ax+By=C$. Then, choose the intervals that contain $A, B, \text{ and } C$.
$$ \text{Equation that was graphed: } -0.75 x - 3 $$ 
The solution is $ 3 x + 4 y = -12 $ 

\begin{enumerate}[label=\Alph*.] 
\item $ A \in [3.23, 4.62], \hspace{3mm} B \in [-3.1, -2.4], \text{ and } \hspace{3mm} C \in [8, 10] $ 

  Corresponds to using the opposite slope of the graph, but did everything else correctly. 
\item $ A \in [-3.51, -2.89], \hspace{3mm} B \in [-4.5, -3.1], \text{ and } \hspace{3mm} C \in [10, 14] $ 

  Corresponds to not making $A$ positive (by multiplying the equation by $-1$). 
\item $ A \in [2.6, 3.72], \hspace{3mm} B \in [2.6, 4.9], \text{ and } \hspace{3mm} C \in [-20, -7] $ 

 * Correct option. 
\item $ A \in [0.26, 1.17], \hspace{3mm} B \in [-0.4, 2.7], \text{ and } \hspace{3mm} C \in [-8, -2] $ 

  Corresponds to not removing rational values. 
\item $ A \in [1.28, 1.95], \hspace{3mm} B \in [-1.9, -0.3], \text{ and } \hspace{3mm} C \in [8, 10] $ 

  Corresponds to using the opposite slope of the graph and not removing rational values. 
\end{enumerate} 
 
General Comments: Standard form is supposed to have $A > 0$ and all fractions removed.

-----------------------------------------------

8. Find the equation of the line described below. Write the linear equation as $ y=mx+b $ and choose the intervals that contain $m$ and $b$.
$$ \text{Parallel to } 7 x + 4 y = 11 \text{ and passing through the point } (-2, 10). $$ 
The solution is $ y = -1.75 x + 6.5 $ 

\begin{enumerate}[label=\Alph*.] 
\item $ m \in [-5, 0] \text{ and } b \in [-9, -5] $ 

  Corresponds to using the correct slope and getting the negative $y$-intercept. 
\item $ m \in [1.2, 2] \text{ and } b \in [12, 16] $ 

  Corresponds to using the negative slope. 
\item $ m \in [-3, -1] \text{ and } b \in [-1, 1] $ 

  Corresponds to using the correct slope and mis-distributing while simplifying to slope-intercept form. 
\item $ m \in [-1.2, -0.4] \text{ and } b \in [5, 9] $ 

  Corresponds to using the reciprocal slope $(1/m)$. 
\item $ m \in [-2.9, -1.1] \text{ and } b \in [3, 8] $ 

 * Correct option. 
\end{enumerate} 
 
General Comments: Parallel slope is the same and perpendicular slope is opposite reciprocal. Opposite reciprocal means flipping the fraction and changing the sign (positive to negative or negative to positive).

-----------------------------------------------

9. Solve the equation below. Then, choose the interval that contains the solution.
$$ -11(9+12 x) = -7(6 x-4) $$ 
The solution is $ -1.411 $ 

\begin{enumerate}[label=\Alph*.] 
\item $ x \in [0.72, 0.84] $ 

  Corresponds to not distributing the negative in front of the first parentheses correctly. 
\item $ x \in [-0.49, -0.23] $ 

  Corresponds to getting the negative of the actual solution. 
\item $ x \in [-1.53, -1.31] $ 

 * Correct option. 
\item $ x \in [-0.97, -0.6] $ 

  Corresponds to not distributing the negative in front of the second parentheses correctly. 
\item $ \text{There are no Real solutions. } $ 

 Corresponds to students thinking a fraction means there is no solution to the equation. 
\end{enumerate} 
 
General Comments: The most common mistake on this question is to not distribute correctly.

-----------------------------------------------

10. Solve the linear equation below. Then, choose the interval that contains the solution.
$$ \frac{5 x + 5}{4} - \frac{4 x - 4}{6} = \frac{5 x - 8}{2} $$ 
The solution is $ 3.087 $ 

\begin{enumerate}[label=\Alph*.] 
\item $ x \in [8.64, 8.9] $ 

  Corresponds to dividing only the first term for each fraction (rather than multiplying to remove the fractions). 
\item $ x \in [2.38, 2.59] $ 

  Corresponds to not distributing the negative correctly for the second fraction. 
\item $ x \in [2.71, 3.23] $ 

 * Correct option. 
\item $ x \in [1.28, 1.65] $ 

  Corresponds to dividing only the second term for each fraction (rather than multiplying to remove the fractions). 
\item $ \text{There are no Real solutions.} $ 

 Corresponds to students thinking a fraction means there is no solution to the equation. 
\end{enumerate} 
 
General Comments: If you are having trouble with this problem, try to remove a fraction at a time by multiplying each term by the denominator.

-----------------------------------------------


\end{document}
