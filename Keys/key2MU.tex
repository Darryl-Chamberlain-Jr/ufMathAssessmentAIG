\documentclass{article}[10pt]
\usepackage{multicol, enumerate, enumitem, hyperref, color, soul, setspace, parskip, fancyhdr, amssymb, amsthm, amsmath, bbm, latexsym, units, mathtools}
\everymath{\displaystyle}
\usepackage[headsep=0.5cm,headheight=0cm, left=1 in,right= 1 in,top= 1 in,bottom= 1 in]{geometry}

\begin{document}
This is the Answer Key for Module 2 Version MU.

6. First, find the equation of the line containing the two points below. Then, write the equation as $ y=mx+b $ and choose the intervals that contain $m$ and $b$.
$$ (8, -8) \text{ and } (-2, -6) $$ 
The solution is $ y = -0.2 x - 6.4 $ 

\begin{enumerate}[label=\Alph*.] 
\item $ m \in [-0.36, 0.11] \text{ and } b \in [-7.11, -6.33] $ 

 * Correct option. 
\item $ m \in [0.1, 0.3] \text{ and } b \in [-5.82, -5.41] $ 

  Corresponds to using the negative slope and the correct equation. 
\item $ m \in [-2, 1] \text{ and } b \in [4.7, 6.77] $ 

  Corresponds to using the correct slope and getting the negative y-intercept. 
\item $ m \in [-1, 3] \text{ and } b \in [-4.51, -2.7] $ 

  Corresponds to using the correct slope/equation but not distributing correctly using the second point. 
\item $ m \in [-1, 1] \text{ and } b \in [-16.77, -15.98] $ 

  Corresponds to using the correct slope/equation but not distributing correctly using the first point. 
\end{enumerate} 
 
General Comments: Remember to keep your points in order when plugging in to the slope formula.

-----------------------------------------------

7. Write the equation of the line in the graph below in the form $Ax+By=C$. Then, choose the intervals that contain $A, B, \text{ and } C$.
$$ \text{Equation that was graphed: } 0.4 x + 2 $$ 
The solution is $ 2 x - 5 y = -10 $ 

\begin{enumerate}[label=\Alph*.] 
\item $ A \in [-2.21, -1.5], \hspace{3mm} B \in [3.63, 6.84], \text{ and } \hspace{3mm} C \in [8, 12] $ 

  Corresponds to not making $A$ positive (by multiplying the equation by $-1$). 
\item $ A \in [0.07, 0.67], \hspace{3mm} B \in [-2.31, 0.99], \text{ and } \hspace{3mm} C \in [-6, 1] $ 

  Corresponds to not removing rational values. 
\item $ A \in [2.44, 2.72], \hspace{3mm} B \in [0.47, 1.08], \text{ and } \hspace{3mm} C \in [3, 8] $ 

  Corresponds to using the opposite slope of the graph and not removing rational values. 
\item $ A \in [4.52, 5.07], \hspace{3mm} B \in [1.36, 2.96], \text{ and } \hspace{3mm} C \in [3, 8] $ 

  Corresponds to using the opposite slope of the graph, but did everything else correctly. 
\item $ A \in [1.49, 2.37], \hspace{3mm} B \in [-6.45, -4.66], \text{ and } \hspace{3mm} C \in [-13, -4] $ 

 * Correct option. 
\end{enumerate} 
 
General Comments: Standard form is supposed to have $A > 0$ and all fractions removed.

-----------------------------------------------

8. Find the equation of the line described below. Write the linear equation as $ y=mx+b $ and choose the intervals that contain $m$ and $b$.
$$ \text{Perpendicular to } 7 x + 8 y = 15 \text{ and passing through the point } (5, 10). $$ 
The solution is $ y = 1.14285714286 x + 4.28571428571 $ 

\begin{enumerate}[label=\Alph*.] 
\item $ m \in [-1, 2] \text{ and } b \in [-6, -2] $ 

  Corresponds to using the correct slope and getting the negative $y$-intercept. 
\item $ m \in [0.95, 1.28] \text{ and } b \in [4, 6] $ 

 * Correct option. 
\item $ m \in [0, 3] \text{ and } b \in [-1, 1] $ 

  Corresponds to using the correct slope and mis-distributing while simplifying to slope-intercept form. 
\item $ m \in [-1.18, -0.97] \text{ and } b \in [14, 18] $ 

  Corresponds to using the negative slope. 
\item $ m \in [0.78, 0.92] \text{ and } b \in [4, 7] $ 

  Corresponds to using the reciprocal slope $(1/m)$. 
\end{enumerate} 
 
General Comments: Parallel slope is the same and perpendicular slope is opposite reciprocal. Opposite reciprocal means flipping the fraction and changing the sign (positive to negative or negative to positive).

-----------------------------------------------


\end{document}
