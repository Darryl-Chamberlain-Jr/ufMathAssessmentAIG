\documentclass{article}[10pt]
\usepackage{multicol, enumerate, enumitem, hyperref, color, soul, setspace, parskip, fancyhdr, amssymb, amsthm, amsmath, bbm, latexsym, units, mathtools}
\everymath{\displaystyle}
\usepackage[headsep=0.5cm,headheight=0cm, left=1 in,right= 1 in,top= 1 in,bottom= 1 in]{geometry}

\begin{document}
This is the Answer Key for Module 2 Version A.

6. First, find the equation of the line containing the two points below. Then, write the equation as $ y=mx+b $ and choose the intervals that contain $m$ and $b$.
$$ (-8, 6) \text{ and } (-2, 9) $$ 
The solution is $ y = 0.5 x + 10.0 $ 

\begin{enumerate}[label=\Alph*.] 
\item $ m \in [0, 3] \text{ and } b \in [13.2, 16.6] $ 

  Corresponds to using the correct slope/equation but not distributing correctly using the first point. 
\item $ m \in [-1, 2] \text{ and } b \in [10.2, 11.9] $ 

  Corresponds to using the correct slope/equation but not distributing correctly using the second point. 
\item $ m \in [-4, 0] \text{ and } b \in [5.7, 8.2] $ 

  Corresponds to using the negative slope and the correct equation. 
\item $ m \in [-1, 2] \text{ and } b \in [-11.1, -8.9] $ 

  Corresponds to using the correct slope and getting the negative y-intercept. 
\item $ m \in [0, 4] \text{ and } b \in [9.3, 10.8] $ 

 * Correct option. 
\end{enumerate} 
 
General Comments: Remember to keep your points in order when plugging in to the slope formula.

-----------------------------------------------

7. Write the equation of the line in the graph below in the form $Ax+By=C$. Then, choose the intervals that contain $A, B, \text{ and } C$.
$$ \text{Equation that was graphed: } 0.75 x + 4 $$ 
The solution is $ 3 x - 4 y = -16 $ 

\begin{enumerate}[label=\Alph*.] 
\item $ A \in [3.83, 4.77], \hspace{3mm} B \in [2.93, 3.78], \text{ and } \hspace{3mm} C \in [10.1, 14] $ 

  Corresponds to using the opposite slope of the graph, but did everything else correctly. 
\item $ A \in [0.47, 1.3], \hspace{3mm} B \in [-1.71, 0.26], \text{ and } \hspace{3mm} C \in [-6.1, -0.5] $ 

  Corresponds to not removing rational values. 
\item $ A \in [2.25, 3.5], \hspace{3mm} B \in [-4.97, -3.63], \text{ and } \hspace{3mm} C \in [-16.9, -14.9] $ 

 * Correct option. 
\item $ A \in [-3.06, -2.71], \hspace{3mm} B \in [3.27, 5.95], \text{ and } \hspace{3mm} C \in [13.9, 16.8] $ 

  Corresponds to not making $A$ positive (by multiplying the equation by $-1$). 
\item $ A \in [0.88, 1.57], \hspace{3mm} B \in [0.51, 1.91], \text{ and } \hspace{3mm} C \in [10.1, 14] $ 

  Corresponds to using the opposite slope of the graph and not removing rational values. 
\end{enumerate} 
 
General Comments: Standard form is supposed to have $A > 0$ and all fractions removed.

-----------------------------------------------

8. Find the equation of the line described below. Write the linear equation as $ y=mx+b $ and choose the intervals that contain $m$ and $b$.
$$ \text{Perpendicular to } 5 x - 9 y = 3 \text{ and passing through the point } (-6, -7). $$ 
The solution is $ y = -1.8 x - 17.8 $ 

\begin{enumerate}[label=\Alph*.] 
\item $ m \in [-4, -1] \text{ and } b \in [-1, 3] $ 

  Corresponds to using the correct slope and mis-distributing while simplifying to slope-intercept form. 
\item $ m \in [1, 2.3] \text{ and } b \in [1, 7] $ 

  Corresponds to using the negative slope. 
\item $ m \in [-4, -1] \text{ and } b \in [16, 23] $ 

  Corresponds to using the correct slope and getting the negative $y$-intercept. 
\item $ m \in [-1.3, 0.4] \text{ and } b \in [-19, -17] $ 

  Corresponds to using the reciprocal slope $(1/m)$. 
\item $ m \in [-2.2, -1.5] \text{ and } b \in [-20, -17] $ 

 * Correct option. 
\end{enumerate} 
 
General Comments: Parallel slope is the same and perpendicular slope is opposite reciprocal. Opposite reciprocal means flipping the fraction and changing the sign (positive to negative or negative to positive).

-----------------------------------------------


\end{document}
