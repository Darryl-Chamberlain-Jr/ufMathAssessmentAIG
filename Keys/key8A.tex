\documentclass{article}[10pt]
\usepackage{multicol, enumerate, enumitem, hyperref, color, soul, setspace, parskip, fancyhdr, amssymb, amsthm, amsmath, bbm, latexsym, units, mathtools}
\everymath{\displaystyle}
\usepackage[headsep=0.5cm,headheight=0cm, left=1 in,right= 1 in,top= 1 in,bottom= 1 in]{geometry}

\begin{document}
This is the Answer Key for Module 8 Version A.

36. Which of the following intervals describes the Range of the function below?
$$ f(x) = \log_2{(x+3)}+9 $$ 
The solution is $ (\infty, \infty) $ 

\begin{enumerate}[label=\Alph*.] 
\item $ (-\infty, a), a \in [-13, -7] $ 

 * Distractor 2: This corresponds to using the using the negative of vertical shift on (0, infinity). 
\item $ [a, \infty), a \in [-4, 0] $ 

 * Distractor 1: This corresponds to using the flipped Domain AND including the endpoint. 
\item $ (-\infty, a], a \in [2, 8] $ 

 * Distractor 4: This corresponds to using the negative of the horizontal shift AND including the endpoint. 
\item $ (-\infty, a), a \in [5, 10] $ 

 * Distractor 3: This corresponds to using the using the vertical shift while the Range is (-infinity, infinity). 
\item $ (-\infty, \infty) $ 

  * This is the correct solution. 
\end{enumerate} 
 
General Comments: The domain of a basic logarithmic function is $(0, \infty)$ and the Range is $(-\infty, \infty)$. We can use shifts when finding the Domain, but the Range will always be all Real numbers.

-----------------------------------------------

37. Which of the following intervals describes the Range of the function below?
$$ f(x) = e^{x+7}+3 $$ 
The solution is $ (3, \infty) $ 

\begin{enumerate}[label=\Alph*.] 
\item $ (-\infty, a], a \in [-3.2, 0.2] $ 

  Distractor 1: This corresponds to using the negative vertical shift AND flipping the Range interval AND including the endpoint. 
\item $ (-\infty, a), a \in [-3.2, 0.2] $ 

  Distractor 2: This corresponds to using the negative vertical shift AND flipping the Range interval. 
\item $ (a, \infty), a \in [2.9, 5.4] $ 

 * This is the solution. 
\item $ [a, \infty), a \in [2.9, 5.4] $ 

  Distractor 3: This corresponds to using the correct vertical shift but including the endpoint. 
\item $ (-\infty, \infty) $ 

 Distractor 4: This corresponds to confusing Range with Domain. 
\end{enumerate} 
 
General Comments: Domain of a basic exponential function is $(-\infty, \infty)$ while the Range is $(0, \infty)$. We can shift these intervals [and even flip when $a<0$!] to find the new Domain/Range.

-----------------------------------------------

38. Solve the equation for $x$ and choose the interval that contains the solution (if it exists).
$$ \log_{3}{(4x+7)}+6 = 3 $$ 
The solution is $ x = -1.741 $ 

\begin{enumerate}[label=\Alph*.] 
\item $ x \in [-4.7, 0.4] $ 

 * This is the solution! 
\item $ x \in [-6.3, -3.4] $ 

  Corresponds to reversing the base and exponent when converting and reversing the value with $x$. 
\item $ x \in [-8.7, -6.1] $ 

  Corresponds to reversing the base and exponent when converting. 
\item $ x \in [3.7, 8.3] $ 

  Corresponds to ignoring the vertical shift when converting to exponential form. 
\item $ \text{There is no Real solution to the equation.} $ 

  Corresponds to believing a negative coefficient within the log equation means there is no Real solution. 
\end{enumerate} 
 
\textbf{General Comments:} First, get the equation in the form $\log_b{(cx+d)} = a$. Then, convert to $b^a = cx+d$ and solve.

-----------------------------------------------

39.  Solve the equation for $x$ and choose the interval that contains $x$ (if it exists).
$$  5 = \ln{\sqrt{\frac{20}{e^x}}} $$ 
The solution is $ x = -7.004000 $ 

\begin{enumerate}[label=\Alph*.] 
\item $ x \in [5,7.8] $ 

  Distractor 1: This corresponds to getting the negative of the solution. 
\item $ x \in [-5.7,-2.7] $ 

  Distractor 2: This corresponds to leaving 1/2 in front of the log. 
\item $ x \in [3.4,6.7] $ 

  Distractor 3: This corresponds to leaving 1/2 in front of the log AND getting the negative of the solution. 
\item $ x \in [-7.4,-4.6] $ 

 * This is the real solution 
\item $ \text{There is no solution to the equation.} $ 

 This corresponds to believing the exponential functional cannot be solved. 
\end{enumerate} 
 
General comments: After using the properties of logarithmic functions to break up the right-hand side, use $\ln(e) = 1$ to reduce the question to a linear function to solve. You can put $\ln(20)$ into a calculator if you are having trouble.

-----------------------------------------------

40. Solve the equation for $x$ and choose the interval that contains the solution (if it exists).
$$ 4^{-3x-3} = \left(\frac{1}{27}\right)^{-2x+4} $$ 
The solution is $ x = 0.839 $ 

\begin{enumerate}[label=\Alph*.] 
\item $ x \in [0.2, 2.4] $ 

 * This is the solution! 
\item $ x \in [-5, -2.3] $ 

  Correponds to ignoring that the bases are different. 
\item $ x \in [-1.5, -0.4] $ 

  Corresponds to getting the negative of the actual solution. 
\item $ x \in [4.1, 4.8] $ 

  Corresponds to ignoring that the basses are different and reversing that solution. 
\item $ \text{There is no Real solution to the equation.} $ 

  Corresponds to believing there is no solution since the bases are not powers of each other. 
\end{enumerate} 
 
\textbf{General Comments:} This question was written so that the bases could not be written the same. You will need to take the log of both sides.

-----------------------------------------------


\end{document}
