\documentclass{article}[10pt]
\usepackage{multicol, enumerate, enumitem, hyperref, color, soul, setspace, parskip, fancyhdr, amssymb, amsthm, amsmath, bbm, latexsym, units, mathtools}
\everymath{\displaystyle}
\usepackage[headsep=0.5cm,headheight=0cm, left=1 in,right= 1 in,top= 1 in,bottom= 1 in]{geometry}

\begin{document}
This is the Answer Key for Module 2 Version B.

6. First, find the equation of the line containing the two points below. Then, write the equation as $ y=mx+b $ and choose the intervals that contain $m$ and $b$.
$$ (-7, 3) \text{ and } (-2, -7) $$ 
The solution is $ y = -2.0 x - 11.0 $ 

\begin{enumerate}[label=\Alph*.] 
\item $ m \in [1, 3] \text{ and } b \in [-3.21, -2.7] $ 

  Corresponds to using the negative slope and the correct equation. 
\item $ m \in [-8, 1] \text{ and } b \in [-5.28, -3.9] $ 

  Corresponds to using the correct slope/equation but not distributing correctly using the second point. 
\item $ m \in [-6, 1] \text{ and } b \in [-11.71, -10.56] $ 

 * Correct option. 
\item $ m \in [-6, 0] \text{ and } b \in [9.47, 10.6] $ 

  Corresponds to using the correct slope/equation but not distributing correctly using the first point. 
\item $ m \in [-5, 0] \text{ and } b \in [10.75, 11.35] $ 

  Corresponds to using the correct slope and getting the negative y-intercept. 
\end{enumerate} 
 
General Comments: Remember to keep your points in order when plugging in to the slope formula.

-----------------------------------------------

7. Write the equation of the line in the graph below in the form $Ax+By=C$. Then, choose the intervals that contain $A, B, \text{ and } C$.
$$ \text{Equation that was graphed: } 1.5 x - 5 $$ 
The solution is $ 3 x - 2 y = 10 $ 

\begin{enumerate}[label=\Alph*.] 
\item $ A \in [0.81, 1.94], \hspace{3mm} B \in [-1.54, -0.08], \text{ and } \hspace{3mm} C \in [4.1, 8] $ 

  Corresponds to not removing rational values. 
\item $ A \in [-0.92, 1.2], \hspace{3mm} B \in [0.72, 1.14], \text{ and } \hspace{3mm} C \in [-18.2, -13.1] $ 

  Corresponds to using the opposite slope of the graph and not removing rational values. 
\item $ A \in [-3.2, -2.25], \hspace{3mm} B \in [1.34, 2.4], \text{ and } \hspace{3mm} C \in [-10.6, -6.2] $ 

  Corresponds to not making $A$ positive (by multiplying the equation by $-1$). 
\item $ A \in [1.84, 2.89], \hspace{3mm} B \in [2.71, 3.29], \text{ and } \hspace{3mm} C \in [-18.2, -13.1] $ 

  Corresponds to using the opposite slope of the graph, but did everything else correctly. 
\item $ A \in [2.99, 3.88], \hspace{3mm} B \in [-2.12, -1.25], \text{ and } \hspace{3mm} C \in [6.1, 13.3] $ 

 * Correct option. 
\end{enumerate} 
 
General Comments: Standard form is supposed to have $A > 0$ and all fractions removed.

-----------------------------------------------

8. Find the equation of the line described below. Write the linear equation as $ y=mx+b $ and choose the intervals that contain $m$ and $b$.
$$ \text{Perpendicular to } 7 x + 4 y = 4 \text{ and passing through the point } (-2, 5). $$ 
The solution is $ y = 0.571428571429 x + 6.14285714286 $ 

\begin{enumerate}[label=\Alph*.] 
\item $ m \in [-1, 2] \text{ and } b \in [-1, 1] $ 

  Corresponds to using the correct slope and mis-distributing while simplifying to slope-intercept form. 
\item $ m \in [-1, -0.1] \text{ and } b \in [3, 6] $ 

  Corresponds to using the negative slope. 
\item $ m \in [-0.2, 1.3] \text{ and } b \in [4, 8] $ 

 * Correct option. 
\item $ m \in [1.6, 2.2] \text{ and } b \in [4, 8] $ 

  Corresponds to using the reciprocal slope $(1/m)$. 
\item $ m \in [0, 3] \text{ and } b \in [-8, -3] $ 

  Corresponds to using the correct slope and getting the negative $y$-intercept. 
\end{enumerate} 
 
General Comments: Parallel slope is the same and perpendicular slope is opposite reciprocal. Opposite reciprocal means flipping the fraction and changing the sign (positive to negative or negative to positive).

-----------------------------------------------

9. Solve the equation below. Then, choose the interval that contains the solution.
$$ -12(-2-14 x) = -15(-6 x+8) $$ 
The solution is $ -1.846 $ 

\begin{enumerate}[label=\Alph*.] 
\item $ x \in [0.37, 0.53] $ 

  Corresponds to getting the negative of the actual solution. 
\item $ x \in [0.44, 0.57] $ 

  Corresponds to not distributing the negative in front of the first parentheses correctly. 
\item $ x \in [-1.96, -1.79] $ 

 * Correct option. 
\item $ x \in [0.93, 1.38] $ 

  Corresponds to not distributing the negative in front of the second parentheses correctly. 
\item $ \text{There are no Real solutions. } $ 

 Corresponds to students thinking a fraction means there is no solution to the equation. 
\end{enumerate} 
 
General Comments: The most common mistake on this question is to not distribute correctly.

-----------------------------------------------

10. Solve the linear equation below. Then, choose the interval that contains the solution.
$$ \frac{6 x + 7}{5} - \frac{8 x - 7}{4} = \frac{-6 x - 8}{3} $$ 
The solution is $ -4.847 $ 

\begin{enumerate}[label=\Alph*.] 
\item $ x \in [-5.67, -3.51] $ 

 * Correct option. 
\item $ x \in [-2.09, -1.65] $ 

  Corresponds to not distributing the negative correctly for the second fraction. 
\item $ x \in [-18.72, -17.11] $ 

  Corresponds to dividing only the first term for each fraction (rather than multiplying to remove the fractions). 
\item $ x \in [-1.92, -1.2] $ 

  Corresponds to dividing only the second term for each fraction (rather than multiplying to remove the fractions). 
\item $ \text{There are no Real solutions.} $ 

 Corresponds to students thinking a fraction means there is no solution to the equation. 
\end{enumerate} 
 
General Comments: If you are having trouble with this problem, try to remove a fraction at a time by multiplying each term by the denominator.

-----------------------------------------------


\end{document}
