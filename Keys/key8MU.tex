\documentclass{article}[10pt]
\usepackage{multicol, enumerate, enumitem, hyperref, color, soul, setspace, parskip, fancyhdr, amssymb, amsthm, amsmath, bbm, latexsym, units, mathtools}
\everymath{\displaystyle}
\usepackage[headsep=0.5cm,headheight=0cm, left=1 in,right= 1 in,top= 1 in,bottom= 1 in]{geometry}

\begin{document}
This is the Answer Key for Module 8 Version MU.

36. Which of the following intervals describes the Domain of the function below?
$$ f(x) = -\log_2{(x+3)}+5 $$ 
The solution is $ (-3, \infty) $ 

\begin{enumerate}[label=\Alph*.] 
\item $ (a, \infty), a \in [-4.22, -1.46] $ 

 * This is the solution. 
\item $ [a, \infty), a \in [-5.89, -4.56] $ 

  Distractor 3: This corresponds to using the negative vertical shift AND including the endpoint. 
\item $ (-\infty, a), a \in [2.14, 3.83] $ 

  Distractor 2: This corresponds to using negative of the horizontal shift. Remember: the general for is a*log(x-h)+k. 
\item $ (-\infty, a], a \in [4.48, 6.12] $ 

  Distractor 1: This corresponds to using the vertical shift when shifting the Domain AND including the endpoint. 
\item $ (-\infty, \infty) $ 

 Distractor 4: This corresponds to thinking of the Range of the log function (or the domain of the exponential function). 
\end{enumerate} 
 
General Comments: The domain of a basic logarithmic function is $(0, \infty)$ and the Range is $(-\infty, \infty)$. We can use shifts when finding the Domain, but the Range will always be all Real numbers.

-----------------------------------------------

37. Which of the following intervals describes the Domain of the function below?
$$ f(x) = -e^{x-5}-3 $$ 
The solution is $ (-\infty, \infty) $ 

\begin{enumerate}[label=\Alph*.] 
\item $ (-\infty, a), a \in [-6, -2] $ 

  Distractor 3: This corresponds to using the correct vertical shift *if we wanted the Range* AND including the endpoint. 
\item $ (a, \infty), a \in [0, 4] $ 

  Distractor 2: This corresponds to using the negative vertical shift AND flipping the Range interval. 
\item $ (-\infty, a], a \in [-6, -2] $ 

  Distractor 4: This corresponds to using the correct vertical shift *if we wanted the Range*. 
\item $ [a, \infty), a \in [0, 4] $ 

  Distractor 1: This corresponds to using the negative vertical shift AND flipping the Range interval AND including the endpoint. 
\item $ (-\infty, \infty) $ 

 * This is the solution. 
\end{enumerate} 
 
General Comments: Domain of a basic exponential function is $(-\infty, \infty)$ while the Range is $(0, \infty)$. We can shift these intervals [and even flip when $a<0$!] to find the new Domain/Range.

-----------------------------------------------

38. Solve the equation for $x$ and choose the interval that contains the solution (if it exists).
$$ \log_{5}{(-4x+7)}+4 = 3 $$ 
The solution is $ x = 1.7 $ 

\begin{enumerate}[label=\Alph*.] 
\item $ x \in [1.91, 2.49] $ 

  Corresponds to reversing the base and exponent when converting. 
\item $ x \in [-29.73, -29.05] $ 

  Corresponds to ignoring the vertical shift when converting to exponential form. 
\item $ x \in [-2.02, -0.88] $ 

  Corresponds to reversing the base and exponent when converting and reversing the value with $x$. 
\item $ x \in [0.79, 1.76] $ 

 * This is the solution! 
\item $ \text{There is no Real solution to the equation.} $ 

  Corresponds to believing a negative coefficient within the log equation means there is no Real solution. 
\end{enumerate} 
 
\textbf{General Comments:} First, get the equation in the form $\log_b{(cx+d)} = a$. Then, convert to $b^a = cx+d$ and solve.

-----------------------------------------------

39.  Solve the equation for $x$ and choose the interval that contains $x$ (if it exists).
$$  7 = \ln{\sqrt{\frac{25}{e^x}}} $$ 
The solution is $ x = -10.781000 $ 

\begin{enumerate}[label=\Alph*.] 
\item $ x \in [-6,-5] $ 

  Distractor 2: This corresponds to leaving 1/2 in front of the log. 
\item $ x \in [2,8] $ 

  Distractor 3: This corresponds to leaving 1/2 in front of the log AND getting the negative of the solution. 
\item $ x \in [-14,-9] $ 

 * This is the real solution 
\item $ x \in [10,13] $ 

  Distractor 1: This corresponds to getting the negative of the solution. 
\item $ \text{There is no solution to the equation.} $ 

 This corresponds to believing the exponential functional cannot be solved. 
\end{enumerate} 
 
General comments: After using the properties of logarithmic functions to break up the right-hand side, use $\ln(e) = 1$ to reduce the question to a linear function to solve. You can put $\ln(25)$ into a calculator if you are having trouble.

-----------------------------------------------

40. Solve the equation for $x$ and choose the interval that contains the solution (if it exists).
$$ 5^{3x+3} = 16^{2x+5} $$ 
The solution is $ x = -12.603 $ 

\begin{enumerate}[label=\Alph*.] 
\item $ x \in [0.2, 1.28] $ 

  Correponds to ignoring that the bases are different. 
\item $ x \in [-2.5, -1.56] $ 

  Corresponds to ignoring that the basses are different and reversing that solution. 
\item $ x \in [-0.46, 0.43] $ 

  Corresponds to getting the negative of the actual solution. 
\item $ x \in [-12.79, -12.25] $ 

 * This is the solution! 
\item $ \text{There is no Real solution to the equation.} $ 

  Corresponds to believing there is no solution since the bases are not powers of each other. 
\end{enumerate} 
 
\textbf{General Comments:} This question was written so that the bases could not be written the same. You will need to take the log of both sides.

-----------------------------------------------


\end{document}
