\documentclass[11pt]{article}
%General Packages
\usepackage{multicol, enumerate, enumitem, hyperref, color, soul, setspace, parskip, fancyhdr}

%Math Packages
\usepackage{amssymb, amsthm, amsmath, bbm, latexsym, units, mathtools}

%All math in Display Style
\everymath{\displaystyle}

%Picture Packages
%\usepackage{etex, graphicx, pstricks, pst-plot, pst-plot, pstricks, pst-node, tikz, pgfplots}
%\reserveinserts{28}

% Packages with additional options
\usepackage[headsep=0.5cm,headheight=0cm, left=1 in,right= 1 in,top= 1 in,bottom= 1 in]{geometry}
\usepackage[usenames,dvipsnames]{xcolor}

% SageTeX
\usepackage{sagetex}

% Package to use the command below to create lines between items
\usepackage{dashrule}
\newcommand{\litem}[1]{\item#1\hspace*{-1cm}\rule{\textwidth}{0.4pt}}

\pagestyle{fancy}
	\lhead{Module 1 - Real and Complex Numbers}
	\chead{}
	\rhead{Progress Exam 3}
	\lfoot{Spring 2019}
	\cfoot{}
	\rfoot{Make-Up Version}

\begin{document}
\pagestyle{fancy}

\begin{sagesilent} 
load("../Code/generalPurposeMethods.sage")
load("../Code/keyGeneration.sage")
\end{sagesilent}

\begin{enumerate}
%OBJECTIVE 1 - Identify the subgroup of Real numbers a number belongs to.
%QUESTION TYPES - Autogenerate a Whole, Integer, Rational, Irrational, or non-real number.
\begin{sagesilent}
version = "MU"
moduleNumber = 1

problemNumber = 1
load("../Code/realComplex/subgroupReal.sage")
\end{sagesilent}
\litem{ \sage{displayStem} 

$$ \sage{displayProblem} $$

	\begin{enumerate}[label=\Alph*.]
		\item $\sage{choices[0]}$
		\item $\sage{choices[1]}$
		\item $\sage{choices[2]}$
		\item $\sage{choices[3]}$
		\item $\sage{choices[4]}$
	\end{enumerate}
	
}

%OBJECTIVE 2 - Apply the properties of Real numbers to simplify large expressions.
\begin{sagesilent}
problemNumber = 2
load("../Code/realComplex/orderOfOperations.sage")
\end{sagesilent}
\litem{ \sage{displayStem}

$$ \sage{displayProblem} $$
\hspace*{10mm} \framebox(30,20){} 
	\begin{enumerate}[label=\Alph*.]
		\item $\sage{choices[0]}$
		\item $\sage{choices[1]}$
		\item $\sage{choices[2]}$
		\item $\sage{choices[3]}$
		\item $\sage{choices[4]}$
	\end{enumerate}

}

%\vspace*{0mm}

%OBJECTIVE 3 - Identify the subgroup of Complex numbers a number belongs to.
%QUESTION TYPES - Autogenerate a Rational, Irrational, Nonreal Complex, Pure Imaginary, or Non-Complex number
\begin{sagesilent}
problemNumber = 3
load("../Code/realComplex/subgroupComplex.sage")
\end{sagesilent}
\litem{ \sage{displayStem}

$$\sage{displayProblem}$$

	\begin{enumerate}[label=\Alph*.]
		\item $\sage{choices[0]}$
		\item $\sage{choices[1]}$
		\item $\sage{choices[2]}$
		\item $\sage{choices[3]}$
		\item $\sage{choices[4]}$
	\end{enumerate}
	
}
\newpage 
%\vspace*{-8mm}
%OBJECTIVE 4 - Add/Subtract/Multiply/Divide Complex numbers.
%TWO questions are drawn from this objective: one multiplication, one division.
\begin{sagesilent}
problemNumber = 4
load("../Code/realComplex/multiplyComplex.sage")
\end{sagesilent}

\litem{ \sage{displayStem}

	$$ \sage{displayProblem} $$
\hspace*{10mm} $a$ = \framebox(30,20){} \hspace*{10mm} $b = $ \framebox(30,20){}
	\begin{enumerate}[label=\Alph*.]
		\item $\sage{choices[0]}$
		\item $\sage{choices[1]}$
		\item $\sage{choices[2]}$
		\item $\sage{choices[3]}$
		\item $\sage{choices[4]}$
	\end{enumerate}
	
}

\begin{sagesilent}
problemNumber = 5
load("../Code/realComplex/divideComplex.sage")
\end{sagesilent}

\litem{ \sage{displayStem}

	$$\sage{displayProblem}$$
\hspace*{10mm} $a$ = \framebox(30,20){} \hspace*{10mm} $b = $ \framebox(30,20){}
	\begin{enumerate}[label=\Alph*.]
		\item $\sage{choices[0]}$
		\item $\sage{choices[1]}$
		\item $\sage{choices[2]}$
		\item $\sage{choices[3]}$
		\item $\sage{choices[4]}$
	\end{enumerate}
	
}

\end{enumerate}

\end{document}
