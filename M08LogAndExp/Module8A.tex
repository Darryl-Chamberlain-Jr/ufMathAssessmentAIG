\documentclass[12pt]{article}
%General Packages
\usepackage{multicol, enumerate, enumitem, hyperref, color, soul, setspace, parskip, fancyhdr}

%Math Packages
\usepackage{amssymb, amsthm, amsmath, bbm, latexsym, units, mathtools}

%All math in Display Style
\everymath{\displaystyle}

% Packages with additional options
%\usepackage[T1]{fontenc}
\usepackage[headsep=0.5cm,headheight=0cm, left=1 in,right= 1 in,top= 1 in,bottom= 1 in]{geometry}
\usepackage[usenames,dvipsnames]{xcolor}

% SageTeX
\usepackage{sagetex}

% Package to use the command below to create lines between items
\usepackage{dashrule}
\newcommand{\litem}[1]{\item#1\hspace*{-1cm}\rule{\textwidth}{0.4pt}}

\pagestyle{fancy}
	\lhead{Module 8 - Logarithmic and Exponential Equations}
	\chead{}
	\rhead{Progress Exam 3}
	\lfoot{Spring 2019}
	\cfoot{}
	\rfoot{Version A}
	

\begin{document}
	\pagestyle{fancy}

\begin{sagesilent} 
load("../Code/generalPurposeMethods.sage")
load("../Code/keyGeneration.sage")
\end{sagesilent}

\begin{enumerate}
\setcounter{enumi}{35}
%OBJECTIVE 1 - Describe the domain/range of Logarithmic and Exponential functions.
\begin{sagesilent}
version = "A"
moduleNumber = 8
problemNumber = 36
load("../Code/logExp/domainRangeLog.sage")
\end{sagesilent}

% TYPE 1 - Describe the domain/range of Logarithmic functions.
\litem{ \sage{displayStem}
$$ \sage{displayProblem} $$
	\begin{enumerate}[label=\Alph*.]
		\item $\sage{choices[0]}$
		\item $\sage{choices[1]}$
		\item $\sage{choices[2]}$
		\item $\sage{choices[3]}$
		\item $\sage{choices[4]}$
	\end{enumerate}		
}

\begin{sagesilent}
problemNumber = 37
load("../Code/logExp/domainRangeExp.sage")
\end{sagesilent}

% TYPE 2 - Describe the domain/range of Exponential functions.
\litem{ \sage{displayStem}
$$ \sage{displayProblem} $$
	\begin{enumerate}[label=\Alph*.]
		\item $\sage{choices[0]}$
		\item $\sage{choices[1]}$
		\item $\sage{choices[2]}$
		\item $\sage{choices[3]}$
		\item $\sage{choices[4]}$
	\end{enumerate}		
}
	
%OBJECTIVE 2 - Covert between logarithmic and exponential forms.
\begin{sagesilent}
problemNumber = 38
load("../Code/logExp/solveByConverting.sage")
\end{sagesilent}
\litem{ \sage{displayStem}
$$ \sage{displayProblem} $$
	\begin{enumerate}[label=\Alph*.]
		\item $\sage{choices[0]}$
		\item $\sage{choices[1]}$
		\item $\sage{choices[2]}$
		\item $\sage{choices[3]}$
		\item $\sage{choices[4]}$
	\end{enumerate}	
}

\newpage 

%OBJECTIVE 3 - Utilize the properties of logaritmic functions to simplify expressions
\begin{sagesilent}
problemNumber = 39
load("../Code/logExp/solveByLogProperties.sage")
\end{sagesilent}

\litem{ \sage{displayStem}
	$$ \sage{a} = \ln \sqrt{\frac{\sage{numerator}}{e^x}} $$
	\begin{enumerate}[label=\Alph*.]
		\item $\sage{choices[0]}$
		\item $\sage{choices[1]}$
		\item $\sage{choices[2]}$
		\item $\sage{choices[3]}$
		\item $\sage{choices[4]}$
	\end{enumerate}	
}

\begin{sagesilent}
problemNumber = 40
load("../Code/logExp/solveExpDifferentBases.sage")
\end{sagesilent}

\litem{ \sage{displayStem}

$$ \sage{displayProblem} $$

\begin{enumerate}[label=\Alph*.]
	\item $\sage{choices[0]}$
	\item $\sage{choices[1]}$
	\item $\sage{choices[2]}$
	\item $\sage{choices[3]}$
	\item $\sage{choices[4]}$
\end{enumerate}		

}

\end{enumerate}

\end{document}